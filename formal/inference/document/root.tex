\documentclass[10pt,DIV16,a4paper,abstract=true,twoside=semi,openright]{scrreprt}
\usepackage[USenglish]{babel}
\usepackage[numbers, sort&compress]{natbib}
\usepackage{isabelle,isabellesym}
\usepackage{booktabs}
\usepackage{paralist}
\usepackage{graphicx}
\usepackage{amssymb}
\usepackage{xspace}
\usepackage{xcolor}
\usepackage{hyperref}
\usepackage{rotating}


\pagestyle{headings}
\isabellestyle{default}
\setcounter{tocdepth}{1}
\newcommand{\ie}{i.\,e.\xspace}
\newcommand{\eg}{e.\,g.\xspace}
\newcommand{\thy}{\isabellecontext}
\renewcommand{\isamarkupsection}[1]{%
  \begingroup%
  \def\isacharunderscore{\textunderscore}%
  \section{#1 (\thy)}%
  \def\isacharunderscore{-}%
  \expandafter\label{sec:\isabellecontext}%
  \endgroup%
}

\newcommand{\orcidID}[1]{} % temp. hack

\newcommand{\repeatisanl}[1]
{\ifnum#1=0\else\isanewline\repeatisanl{\numexpr#1-1}\fi}
\newcommand{\snip}[4]{\repeatisanl#2#4\repeatisanl#3}

\title{A Formal Model of Extended Finite State Machine}%
\author{Michael~Foster\orcidID{0000-0001-8233-9873} \and
 Ramsay~G.~Taylor\orcidID{0000-0002-4036-7590} \and
 Achim~D.~Brucker\orcidID{0000-0002-6355-1200} \and
 John~Derrick\orcidID{0000-0002-6631-8914}}
\publishers{
  Department of Computer Science\\
  The University of Sheffield\\
  Sheffield, UK\\
  \texttt{\{%
	\href{mailto:jmafoster1@sheffield.ac.uk}{jmafoster1},
	\href{mailto:a.brucker@sheffield.ac.uk}{a.brucker},
	\href{mailto:r.g.taylor@sheffield.ac.uk}{r.g.taylor},
	\href{mailto:j.derrick@sheffield.ac.uk}{j.derrick}
  \}@sheffield.ac.uk}
}

\begin{document}
\maketitle
\begin{abstract}
  This theory formalises a technique based on state merging to infer EFSM models from black box traces, as presented in \cite{foster2019}. A Code Generator setup to create executable Scala code is also defined.

  \begin{quote}
    \bigskip
    \noindent{\textbf{Keywords:}EFSM inference, Model inference, Reverse engineering }
  \end{quote}
\end{abstract}


\tableofcontents
\cleardoublepage

\chapter{Introduction}
This theory formalises the EFSM inference technique from \cite{foster2019}. The technique builds off classical FSM inference techniques which work by merging states which share behaviour. The process first builds a Prefix Tree Acceptor from the traces, and then states are iteratively merged to form a smaller model.

The rest of this document is automatically generated from the formalization in Isabelle/HOL, i.e., all content is checked by Isabelle.  Overall, the structure of this document follows the theory dependencies (see \autoref{fig:session-graph}).

\begin{sidewaysfigure}
  \centering
  \resizebox{\textheight}{!}{\includegraphics[height=\textheight]{session_graph}}
  \caption{The Dependency Graph of the Isabelle Theories.\label{fig:session-graph}}
\end{sidewaysfigure}

\nocite{foster.ea:efsm:2018}

\clearpage
\chapter{Theories}

\input{session}


{\small
  \bibliographystyle{abbrvnat}
  \bibliography{root}
}
\end{document}
\endinput

%%% Local Variables:
%%% mode: latex
%%% TeX-master: t
%%% End:
